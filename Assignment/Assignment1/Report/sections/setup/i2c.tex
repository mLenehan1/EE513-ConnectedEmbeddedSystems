In order to setup I2C on the Raspberry Pi 3 Model B+, the I2C bus must first be
enabled. This is achieved by executing the command ``raspi-config''. Navigating
to the BLANK tab, and selecting BLANK to enable the bus.

Once the circuit is connected, the i2ctools package must be installed in order
to verify the connection. The package can be installed by executing the
following command:

\begin{lstlisting}[language=bash]
	sudo apt install i2c-tools
\end{lstlisting}

Once installed the I2C bus can be scanned using the ``i2cdetect command''. The
``-l'' option lists the installed busses.

\begin{lstlisting}[language=bash]
	i2cdetect -l
\end{lstlisting}

The output of this command corresponds to the I2C bus that the device is
connected to. Using the ``-r'' option, bus 1 can be scanned using
the SMBus ``receive byte'' method. The ``-y'' option in this case disables user
input for confirmation.

\begin{lstlisting}[language=bash]
	i2cdetect -y -r 1
\end{lstlisting}

Finally the ``i2cdump'' command can be used to dump the contents of the
specified address, in this case ``0x68'' on bus 1:

\begin{lstlisting}[language=bash]
	i2cdump -y 1 0x68
\end{lstlisting}
