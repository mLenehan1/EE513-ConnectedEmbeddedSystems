For the novel functionality, it was decided that it would be a useful feature to
be able to initialize the RTC time based on the current system time. This was
done using a combination of file I/O operations, and a system call to the Linux
``date'' utility. This utility allows for the formatting of the output based on
a number of input arguments. In this case the arguments ``+\%S \%M \%H \%u \%d \%m
\%Y'' specify to output the seconds, minutes, hours, weekday number, date, month,
and year, delimited by spaces.

The output of the ``date'' command is passed into a temporary file, ``tmp'' from
which it can be read, with the symbols delimited by the whitespace. The values
are placed in a string vector via the ``push\_back'' method. The temporary file
is then closed and deleted. The vector values are converted to integers using
the ``stoi'' function, and placed in separate time and date character arrays,
from which they can be passed into the ``writeTime'' and ``writeDate''
functions.

\lstinputlisting[language=C++, caption={Novel Function - Initialize RTC time
from System Time}, label={lst:novel}]{snippets/novel.cpp}
