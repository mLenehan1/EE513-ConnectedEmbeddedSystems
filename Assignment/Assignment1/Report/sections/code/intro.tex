In order to develop C++ code for the Raspberry Pi, the code must be compiled via
an ARM compatible C++ compiler. This can be achieved in a number of ways,
including cross-compilation and compilation on the target hardware (i.e. the
Raspberry Pi). As the Raspberry Pi is using the Raspbian ``Stretch'' image,
there is no graphical user environment, i.e. no window manager, or desktop
environment. As such, remote development was used to develop on the Raspberry
Pi. The ``Visual Studio Code'' IDE has the ability to connect to a host via SSH,
allowing a developer to work on the targets filesystem within the IDE. A
terminal can also be opened from within the IDE, allowing terminal commands,
including compilation commands, to be run from within the IDE.

Using the provided C code, an I2C base class was created. This class contains
all of the generic I2C code required for opening I2C bus connections, device
connections, setting read addresses, and reading from registers.

A DS3231 class, which is a child of the I2CDevice class, has a number of methods
which are specific to the DS3231. These include all of the required methods for
reading and writing time, date, alarm, interrupt, and temperature data to the
device. All of the code used for this assignment can be found within the
appendices.
