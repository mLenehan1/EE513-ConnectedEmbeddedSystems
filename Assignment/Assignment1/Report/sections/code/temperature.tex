The temperature value can be read from the RTC in much the same way as the time,
however, the values within the temperature registers are stored in two's
compliment representation, with the upper two bits of the least significant bits
register representing a ratio of the degree value. With a precision of $\pm$
0.25 degrees, the LSB bits can represent a 0.00 degrees (bits 00), 0.25 degrees
(bits 01), 0.50 degrees (bits 10), and 0.75 degrees (bits 11).

\lstinputlisting[languag=C++, caption={readTemp Function}]{snippets/temp.cpp}

In order to extract the upper two bits of a byte, a bitwise right shift by 6
places is used. Multiplying 100 by the new value and dividing by 4 gives the
degrees ratio value. This can then be printed to the screen. The temperature MSB
value is stored in the register indexed `17' with the LSB in the register
indexed `18'.
